\chapter*{Introduction}
\addcontentsline{toc}{chapter}{Introduction}

Artificial Intelligence (AI) is a fast-growing field of computer science that focuses on the creation of intelligent machines that can simulate human cognition. In recent years, AI technology has been applied in a wide range of fields, including healthcare, finance, and transportation, with the goal of improving efficiency, accuracy, and decision-making. To gain insights into the capabilities and limitations of AI algorithms and techniques, many researchers turned to games as a testing platform to evaluate and compare different methods as they provide a convenient and controllable environment to achieve the aforementioned goals. 

In recent decades, computer games have also gained popularity, similar to the growth of AI as a field of study. Due to its utility, the game industry has become one of the many fields that have sought to use AI to their advantage. Being a subject of extensive research, perfect information in two-player games has been a common focus in game theory, which allowed the development of algorithms for a greater understanding of games. However, in a manner similar to the real world, situations in which all relevant information is available are not always present. Given the inherent 
characteristics of their environment, the design of algorithms for imperfect information games is more challenging. Therefore, this thesis seeks to contribute to this field by developing algorithms for the game "Durak".

Durak is a strategic card game that originated in Russia. It is played with a deck of cards and typically involved two to six players. Unlike the other games, the aim of the Durak is not to find a winner, but to find a loser. Players take turns attacking and defending in a series of rounds. During an attack, the attacking player leads with one or more cards, and the defending player must attempt to beat them by playing a higher-ranked card. If the defending player is unable or unwilling to do so, they must pick up all the cards. The goal of the game is to get rid of all of one's cards, and the player left holding cards at the end is declared the fool. \cite{website:PAGAT}

Given the intricate nature of the game, a key objective is to ensure its correct development with all relevant details. As the game will include various AI agents, it is essential for the game model to provide a suitable interface for the integration of AI agents. 

Another goal of this thesis is to implement a range of AI players for the given game model. One of the benefits of introducing the agents for this game is that it will provide an opportunity to examine potential challenges associated with implementing AI for games of this type, as well as verify the suitability and usability of the game's API for this purpose.

After implementing the AI agents, the aim is to compare their performance in a mutual play, with the objective of identifying the most effective technique. The AI players must not only win against all other agents but must also make moves quickly, ideally at least several moves per second on average. This requirement reflects the need for AI players to be both effective and efficient in their decision-making. This comparison will provide valuable insights into the strengths and weaknesses of the various AI approaches and will help to guide future work in this area. 
