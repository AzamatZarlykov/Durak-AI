\chapter{Game Analysis}

As described in the chapter \ref{GameDescription}, Durak is a multifunctional game that requires players to consider a range of factors in order to play effectively. This complexity is a key characteristic of the game, and contributes to its strategic depth and appeal. Given its intricate nature, this chapter will analyze the game from the game-theoretic perspective in order to understand its underlying structure and strategic considerations. This will involve categorizing the game according to relevant criteria and examining the complexity of the game as a whole.

\section{Classification}

Durak can be classified as a \textbf{discrete game}. A discrete game is a type of game in which players have a finite number of choices, or actions, that they can take \citep{Gametheory4}. This applies in Durak. Players have a limited number of choices that they can make at each turn. They can choose which card to play, and must decide whether to attack or defend. These choices are limited by the cards that the player has in their hand and the rules of the game. 

Furthermore, it can be considered a \textbf{sequential} game from a game-theoretic perspective. A sequential game is a type of game in which the order in which players make their decisions matters \citep{Gametheory4}. In Durak, the order in which players play their cards is important, as it determines who is able to attack and who must defend. The sequence of actions is determined by the rules of the game described in chapter \ref{GameDescription}, and players must consider the potential actions of their opponents as they make their own decisions. 

In addition, Durak can be classified as a game of \textbf{imperfect information}. In a game of imperfect information, players do not have complete information about the game state or the actions of their opponents \citep{Gametheory4}. They must make decisions based on incomplete information and must try to infer the actions of their opponents based on their observations and past experiences. As described before, Durak is a game of imperfect information because players do not have complete information about the cards in the hands of their opponents. They must make decisions about which cards to play and when to use their trump cards based on incomplete information, and must adapt their strategies as the game progresses and new information becomes available.

Additionally, Durak can be classified as a \textbf{non-deterministic} game due to the presence of elements of randomness that can affect the outcome of the game. In game theory, a non-deterministic game is a type of game in which the outcomes are not determined solely by the actions of the players and the rules of the game, but are also influenced by random events or factors \citep{Gametheory4}. In case of the game, the distribution of cards at the beginning and the order in which cards are played can both be considered elements of randomness that can affect the outcome of the game.

In summary, Durak can be classified as a discrete, sequential, imperfect information, and non-deterministic game from a game-theoretic perspective, which contribute to its complexity and strategic depth.


\section{Branching Factor}

The branching factor of a game refers to the number of possible moves that a player can make at each turn. In ``Podkidnoy Durak'', it can be challenging to determine the branching factor as it varies depending on the specific game state. At each turn, the number of possible moves a player can make is influenced by the cards in their hand and the cards on the table, as well as the defending or attacking rules.