\chapter{Game Analysis}

As described in the chapter \ref{GameDescription}, Durak is a game that requires players to consider a range of factors in order to play effectively. Given its intricate nature, this chapter will analyze the game from the game-theoretic perspective in order to understand its underlying structure and strategic considerations. This will involve categorizing the game according to relevant criteria, examining the complexity of the game as a whole, comparing the length of the game and introducing new concept.

Prior to discussing the fundamental structure and strategic elements of Durak, it is essential to acknowledge that every game of Durak must eventually conclude, as it is not possible for a game to continue indefinitely. For a game to persist, players would need to repeatedly exchange the same set of cards. However, this scenario is not feasible because cards cannot return to a previous owner until certain cards from the bout are placed in the discard pile. The inclusion of cards in the discard pile allows for the changing of turns, enabling the return of previously exchanged cards to their original owner. As the exchange of cards between players continues, additional cards from the deck and ultimately from the players' hands will be placed in the discard pile until only the cards being exchanged remain.

Another question of interest is whether it is possible for the same game state to occur twice within a single game of Durak. Despite the numerous exchanges of cards between players, it is not possible for the same game state to be replicated. For the same state to repeat, players would need to exchange the same set of cards in a circle, such as player A attacking with an Ace card and player B taking it, then player B attacking with the same Ace on the next turn to player A. However, this scenario is not feasible. As previously mentioned, the intermediate bout must come to an end and cards from it must be placed in the discard pile in order to allow player B to return the Ace to player A. By this point, the game state will have changed due to the presence of additional cards in the discard pile that were not present when the Ace belonged to player A. As a result, it is not possible for the same game state to repeat in a single game of Durak.

Also, it is a common misconception that the first player to move in a game of Durak holds a significant advantage over their opponent. While it is true that the first player has the opportunity to set the tone of the game and establish their strategy from the outset, this advantage is not necessarily decisive. A skilled opponent can effectively counter the strategies of the first player and ultimately achieve victory. The outcome of a game of Durak is determined by the players' abilities and strategies, rather than the order in which they move.

\section{Classification}

Durak can be classified as a \textbf{discrete game}. A discrete game is a type of game in which players have a finite number of choices, or actions, that they can take \citep{Gametheory4}. This applies in Durak. Players have a limited number of choices that they can make at each turn. They can choose which card to play, and must decide whether to attack or defend. These choices are limited by the cards that the player has in their hand and the rules of the game. 

Furthermore, it can be considered a \textbf{sequential} game from a game-theoretic perspective. A sequential game is a type of game in which the order in which players make their decisions matters \citep{Gametheory4}. In Durak, the order in which players play their cards is important, as it determines who is able to attack and who must defend. The sequence of actions is determined by the rules of the game described in chapter \ref{GameDescription}, and players must consider the potential actions of their opponents as they make their own decisions. 

In addition, Durak can be classified as a game of \textbf{imperfect information}. In a game of imperfect information, players do not have complete information about the game state or the actions of their opponents \citep{Gametheory4}. They must make decisions based on incomplete information and must try to infer the actions of their opponents based on their observations and past experiences. As described before, Durak is a game of imperfect information because players do not have complete information about the cards in the hands of their opponents. They must make decisions about which cards to play and when to use their trump cards based on incomplete information, and must adapt their strategies as the game progresses and new information becomes available.

Additionally, Durak is often played in a \textbf{deterministic} manner, meaning that the outcome of the game is determined by the initial cards that are dealt and the actions that are taken by the players during the game. However, there is some element of chance in Durak, as the cards are shuffled randomly before the game begins can affect the outcome of the game. Therefore, it is possible to consider Durak to be a \textbf{non-deterministic} game to some degree. In game theory, a non-deterministic game is a type of game in which the outcomes are not determined solely by the actions of the players and the rules of the game, but are also influenced by random events or factors \citep{Gametheory4}. 

In summary, Durak can be classified as a discrete, sequential, imperfect information, and non-deterministic to some extent game from a game-theoretic perspective, which contribute to its complexity and strategic depth.


\section{Branching Factor}

The branching factor of a game refers to the number of possible moves that a player can make at each turn. In ``Podkidnoy Durak'', it can be challenging to determine the branching factor as it varies depending on the specific game state. At each turn, the number of possible moves a player can make is influenced by the cards in their hand and the cards on the table, as well as the defending or attacking rules. To be specific, the attacker can initiate an attack by playing any card from their hand. Therefore, the maximum branching factor is the maximum number of cards that a player can hold in their hand, which is 35 if one player holds all but one of the cards. However, there are also situations in which a player may only have one possible move, such as when they are unable to defend against an attack and must pass and take the card. Therefore, the average branching factor in this game is relatively low, as players often have only a few choices of cards to play in a given situation. This is particularly true for the defender, who may only have a few options for defending against an attack, and for subsequent attacks, where the number of available options may also be limited.

To clarify the branching factor assumption, I have run an experiment to estimate the average branching factor in Durak by simulating 1000 random games played between two greedy agents. The results showed that the average branching factor, as computed using the \textbf{geometric mean}, was \textbf{2.17}, which suggests that the branching factor of the game is low. 

\section{Duration}

The objective of this section is to determine the typical duration of games of Durak in terms of bouts and plies, where ply refers to a single move made by a single player. To address this question, we will compare the average length of games played between two random players and two greedy agents, in order to examine the influence of player strategy on the duration of the game.

To compare the durations of player strategies in Durak, we conducted two experiments. The first experiment involved 1000 games played between two greedy agents, and the second experiment involved 1000 games played between two random agents. The results of the first experiment showed that the average number of bouts per game was 8.3, the average number of plies per bout was 5.3, and the average number of plies per game was 44.0. The results of the second experiment showed that the average number of bouts per game was 24.0, the average number of plies per bout was 2.9, and the average number of plies per game was 68.0. These findings suggest that the behavior of the random agents led to longer games, as evidenced by the higher number of bouts and lower number of plies per bout in the second experiment.

\section{Weakness Concept}
\label{weaknessConcept}
This section introduces the concept of \textbf{weakness} in Durak, which is a concept that arises from the analysis of the game and may be relevant to various strategies or agents.

Edouard Bonnet's paper, 'The Complexity of Playing Durak,' examines the difficulty of identifying winning strategies in the card game Durak. Bonnet's work demonstrates that, even in a perfect information setting with two players, finding optimal moves is a challenging computational problem. Specifically, Bonnet establishes that determining the presence of a winning strategy in a generalized Durak position is PSPACE-complete. In my own research, I aim to construct a strong agent capable of playing optimally in both perfect and imperfect information settings. Bonnet's contributions, including the concept of weaknesses, have been invaluable in the development of my rule-based agent.

A weakness for a player, referred to as player P, is defined as a rank r that meets the following criteria: 
\begin{enumerate}
	\item player P's hand contains at least one card of rank r, and
	\item for each suit s of rank r in player P's hand, there exists a rank $r$'$ > r$ such that the opponent holds a card of rank r$'$ and suit s \citep{Bonnet2016TheCO}.
\end{enumerate}

To clarify the concept of weaknesses, consider the following scenario: Player P holds the cards 10$\textcolor{red}{\heartsuit}$, 10$\textcolor{black}{\spadesuit}$, and K$\textcolor{red}{\diamondsuit}$, while player O holds Q$\textcolor{red}{\heartsuit}$, Q$\textcolor{black}{\spadesuit}$, and J$\textcolor{black}{\clubsuit}$. In this case, player P has a weakness at rank 10, as it satisfies the two conditions outlined in the definition of weakness. Specifically, player P holds at least one card of rank 10 (10$\textcolor{red}{\heartsuit}$ and 10$\textcolor{black}{\spadesuit}$), and for each suit of rank 10 in player P's hand (10$\textcolor{red}{\heartsuit}$ and 10$\textcolor{black}{\spadesuit}$), the opponent holds a card of higher rank (Q$\textcolor{red}{\heartsuit}$ and Q$\textcolor{black}{\spadesuit}$, respectively).

